\documentclass[12pt,a4paper]{article}

\makeatletter
    \input{../config/header[fr].sty}

    \usepackage{01-matrices}
\makeatother



\begin{document}

\section{Matrices}

Tout le boulot ou presque est fait par l'excellent package \verb+nicematrix+
\footnote{
	On impose l'option \texttt{transparent}.
} 
avec l'ajout 
d'une macro
%de deux macros
\emph{\og maison \fg} à but pédagogique.
Veuillez vous reporter à la documentation de \verb+nicematrix+ pour savoir comment s'y prendre en général.


% ---------------------- %


\subsection{Quelques exemples pour bien démarrer}

\newparaexample{Vu dans la documentation de \texttt{nicematrix}}

\begin{latexex}
$\begin{pmatrix}
    1      & \cdots & \cdots & 1      \\
    0      & \ddots &        & \vdots \\
    \vdots & \ddots & \ddots & \vdots \\
    0      & \cdots & 0      & 1
\end{pmatrix}$
\end{latexex}


% ---------------------- %


\newparaexample{}

\begin{latexex}
$\begin{vmatrix}
    1      & \cdots & \cdots & 1      \\
    0      & \ddots &        & \vdots \\
    \vdots & \ddots & \ddots & \vdots \\
    0      & \cdots & 0      & 1
\end{vmatrix}$
\end{latexex}


% ---------------------- %


\newparaexample{}

\begin{latexex}
$\begin{bmatrix}
    1      & \cdots & \cdots & 1      \\
    0      & \ddots &        & \vdots \\
    \vdots & \ddots & \ddots & \vdots \\
    0      & \cdots & 0      & 1
\end{bmatrix}$
\end{latexex}


% ---------------------- %


\newparaexample{Vu dans la documentation de \texttt{nicematrix}}

\begin{latexex}
$\begin{pNiceMatrix}[name = mymatrix]
     1 & 2 & 3 \\
     4 & 5 & 6 \\
     7 & 8 & 9
 \end{pNiceMatrix}$

 \tikz[remember picture,
       overlay]
 \draw[red]
     (mymatrix-2-2) circle (2.5mm);
\end{latexex}


% ---------------------- %


\newparaexample{Vu dans la documentation de \texttt{nicematrix}}

\begin{latexex}
$\left(
     \begin{NiceArray}{CCCC:C}
         1  & 2  & 3  & 4  & 5  \\
         6  & 7  & 8  & 9  & 10 \\
         11 & 12 & 13 & 14 & 15
     \end{NiceArray}
 \right)$
\end{latexex}$


% ---------------------- %


\newparaexample{Proposition de l'auteur de \texttt{nicematrix} suite à une discussion par mail}

\begin{latexex}
% Besoin du package ``ifthen``.
\newcommand\aij{%
  a_{\arabic{iRow}\arabic{jCol}}%
}

$\begin{bNiceArray}{*{5}{>{%
     \ifthenelse{\value{iRow}>0}{\aij}{}%
 }C}}[
     first-col,
     first-row,
     code-for-first-row
       = \mathbf{\arabic{jCol}},
     code-for-first-col
       = \mathbf{\arabic{iRow}}
   ]
       & & & & & \\
       & & & & & \\
       & & & & &
 \end{bNiceArray}$
\end{latexex}$


% ---------------------- %


\newparaexample{Avec des calculs automatiques}

\begin{latexex}
\newcounter{cntaij}
\newcommand\aij{%
    \setcounter{cntaij}{\value{iRow}}%
    \addtocounter{cntaij}{\value{jCol}}%
    \addtocounter{cntaij}{-1}%
    \arabic{cntaij}%
}

Si $a_{ij} = i + j - 1$ alors

$(a_{ij})_{1 \leq i \leq 3 ,
           1 \leq j \leq 5}
 =
 \begin{bNiceArray}{*{5}{>{\aij}C}}
     & & & & \\
     & & & & \\
     & & & &
 \end{bNiceArray}$
\end{latexex}

\end{document}
